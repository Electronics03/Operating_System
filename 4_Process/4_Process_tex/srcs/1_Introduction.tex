\section{Introduction}
The operating system acts as an interface between hardware and the user.
It manages resources to ensure that programs run efficiently and safely,
and provides abstractions to simplify complex hardware operations.

\subsection{CPU Virtualization}
Physically, a CPU can only execute one instruction at a time.
However, users perceive that multiple programs are running simultaneously.
This illusion is created because the operating system uses CPU Virtualization,
dividing CPU time among processes.
The OS achieves concurrency through time sharing and scheduling.

\subsection{API}
There are two general ways for a user program to control hardware: directly or indirectly.
If hardware were controlled directly, the OS could lose control of the system
whenever a program failed to yield resources.
Therefore, the indirect method is preferred.
However, user programs sometimes need direct access to hardware.
To safely provide this functionality, the OS exposes
an Application Programming Interface (API) to interact with hardware.

\subsection{System Call}
User programs can only access resources through system calls provided by the OS.
System calls are a type of API that provide controlled access to critical functions,
such as reading/writing files, creating/terminating processes, and memory allocation.
This mechanism allows the OS to maintain security and stability
while enabling applications to use hardware functionality.