
\section{Process}
A process is one of the most important abstractions provided by the OS,
and is generally defined as a running program.
A program is just a collection of instructions and static data on disk,
but when the OS loads it into memory and executes it,
it becomes an active process.

Users want to run multiple applications simultaneously.
Even though modern systems usually have only a few CPUs,
the OS uses CPU Virtualization to create the illusion
that dozens or hundreds of processes can run at the same time.
By applying time sharing, the OS executes one process briefly, suspends it,
and switches to another, repeating this cycle to provide the illusion of concurrency.

\subsection{Concept of Process}
A process is essential to CPU Virtualization.
It consists of the following elements:
\begin{itemize}
    \item \textbf{Memory space:} code, static data, heap, stack
    \item \textbf{Registers:} PC, stack pointer.
    \item \textbf{Open file list:} including standard input, output, and error streams
\end{itemize}

A process can be in one of three states:
\begin{itemize}
    \item \textbf{Running:} currently executing on the CPU
    \item \textbf{Ready:} prepared to run, but waiting for CPU allocation
    \item \textbf{Blocked:} waiting on an event such as I/O, unable to execute
\end{itemize}

\subsection{Process Control Block (PCB)}
The OS tracks each process via a Process Control Block (PCB), which includes:
\begin{itemize}
    \item PID and process state (Running/Ready/Blocked)
    \item Saved register context (for context switching)
    \item Memory management info (code/data/heap/stack regions)
    \item Open file descriptors and current working directory
    \item Parent/child relations and pending signals
\end{itemize}

\subsection{Process API}
Modern operating systems provide the following APIs to control processes:
\begin{itemize}
    \item \textbf{Create:} create and start a new process
    \item \textbf{Destroy:} terminate a running process
    \item \textbf{Wait:} wait until a specific process finishes
    \item \textbf{Miscellaneous Control:} suspend, resume, or otherwise control processes
    \item \textbf{Status:} retrieve execution time, state, or other process information
\end{itemize}

\subsection{Process Creation}
The transition from program to process consists of several steps:
\begin{enumerate}
    \item Load the program code and static data into memory.
    \item Initialize the runtime stack, including \texttt{argc} and \texttt{argv[]}.
    \item Create a heap region for dynamic memory allocation.
    \item Initialize file descriptors for standard input, output, and error.
    \item Begin execution at the program entry point, \texttt{main()}.
\end{enumerate}